\documentclass[a4paper,uplatex]{jsarticle}
\begin{document}
\newcommand{\myouji}{川崎}%ここは\def\myouji{川崎} でもOK!
\newcommand{\midashi}[1]{\vskip13pt\noindent{\Large\gtfamily #1}}%ここは\def\midashi#1{\vskip13pt\noindent{\Large\gtfamily #1}}

自己紹介をします。2015年東大入学なので,現在学部の5年生です。\myouji ほげお(仮名)といいます。

\midashi{\myouji ほげおの趣味}
\begin{itemize}
\item 落語鑑賞
\item カラオケ
\item 水泳
\item 生徒の質問に答えること
\end{itemize}

\midashi{\myouji ほげおのサークル}
\begin{itemize}
\item 東大\TeX 愛好会
\item 東大SATySFi愛好会
\end{itemize}

\midashi{\myouji ほげおの生息場所}
\begin{itemize}
\item 代々木
\item 本郷
\item 駒場
\item 川崎市
\end{itemize}

\midashi{授業内容}

$\left[\mathrm{CH_3COOH}\right]$,$\left[\mathrm{CH_3COO^-}\right]$,$\left[\mathrm{H^+}\right]$に対して$K_{\mathrm{a(CH_3COOH)}}$は
\[
K_{\mathrm{a(CH_3COOH)}}=\frac{\left[\mathrm{CH_3COO^-}\right]\left[\mathrm{H^+}\right]}{\left[\mathrm{CH_3COOH}\right]}
\]
と定義される。


\end{document}